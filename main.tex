
% ======================================================================
% starting package maintenance...
% installation directory: "C:\Users\Bernardo Goncalves\AppData\Local\Programs\MiKTeX"
% package repository: https://mirror.kku.ac.th/CTAN/systems/win32/miktex/tm/packages/
% package repository digest: bc26c9dcac319ee10e5af84d3734de2b
% going to download 69253 bytes
% going to install 43 file(s) (1 package(s))
% downloading https://mirror.kku.ac.th/CTAN/systems/win32/miktex/tm/packages/latexindent.tar.lzma...
% 0.07 MB, 0.24 Mbit/s
% extracting files from latexindent.tar.lzma...
% ======================================================================
\documentclass{article}

% Language setting
% Replace `english' with e.g. `spanish' to change the document language
\usepackage[english]{babel}

% Set page size and margins
% Replace `letterpaper' with `a4paper' for UK/EU standard size
\usepackage[a4paper,top=2cm,bottom=2cm,left=3cm,right=3cm,marginparwidth=1.75cm]{geometry}

% Useful packages
\usepackage{amsmath}
\usepackage{graphicx}
\usepackage{subcaption}
\usepackage{booktabs}
\usepackage{multirow}
\usepackage{array}
\usepackage[nottoc]{tocbibind}

% \usepackage[backend=biber,style=numeric]{biblatex}
% \addbibresource{adrenal.bib}

\usepackage[
    backend=bibtex,
    style=ieee,
  ]{biblatex}
\addbibresource{adrenal.bib}

% \usepackage{natbib}
% \bibliographystyle{ksfh_nat}

\usepackage[colorlinks=true, allcolors=blue]{hyperref}

\title{Detecting adrenal lesions using machine learning - a state of the art}
\author{Bernardo Gonçalves - 58885 - Doctoral program in Biomedical Engineering}

\begin{document}
\maketitle

\section{Adrenal Glands and Adrenal Lesions}

\subsection{Anatomy, physiology and pathophysiology (CHANGE)}

The adrenal glands, or suprarenal glands, are two small glands located on top of
the kidneys.  Each has a body and two limbs \cite{Baba2012}, and weights about
5g \cite{brit}. Figure \ref{fig:adrenal_ana} shows the localisation and anatomy
of the adrenal glands. These glands are a component of the
Hypothalamic-Pituitary-Adrenal (HPA) axis, which is responsible to maintain
homeostasis in the presence of chronic stressors, activating a complex range of
responses from the endocrine, nervous and immune systems, generally known as the
stress response \cite{open}.

\begin{figure}
    \centering
    \includegraphics[scale=0.8]{figures/adrenal_anatomy_cropped.png}
    \caption{Adrenal glands localisation and anatomy.  Adapted from \cite{Grossman2022}}
    \label{fig:adrenal_ana}
\end{figure}

The adrenal glands can be affected by a wide variety of benign and malignant
lesions. It is estimated that approximately 9\% of the global population has
adrenal lesions, which are mostly incidentally detected during abdominal imaging
\cite{Dhamija2015}. These lesions can be primary if they originated in the
glands themselves (cortex or medulla) or secondary if they have another origin.
Primary lesions can be functional if they produce hormones \cite{Panda2015}.
Table \ref{tab:lesions} presents the adrenal lesions and their classification.
The most recurrent adrenal lesions are adenomas. Adenomas are often
non-functional and remain asymptomatic, being discovered incidentally
\cite{Platzek2019}. Adrenocortical carcinoma is a rare lesion, although it is
the most common primary malignant adrenal lesion. This lesion affects children
in their first decade and adults in their fourth and fifth decades
\cite{Panda2015}. Also, the adrenals are a frequent location of metastases
\cite{Platzek2019}.

\begin{table}[]
    \centering
    \begin{tabular}{cccc}\toprule
        \textbf{Adrenal Lesions}        & \textbf{Histologic origin}          & \textbf{Benign}       & \textbf{Malign}      \\
        \midrule
        \multirow{6}{*}{Primary}        & \multirow{4}{*}{Cortical}           & Adenoma               & Adrenocortical       \\
                                        &                                     & Adrenal hyperplasia   & Carcinoma            \\
                                        &                                     & Oncocytoma            & Malignant            \\
                                        &                                     &                       & oncocytoma           \\
        \cmidrule(lr){2-4}
                                        & \multirow{4}{*}{Medullary}          & Pheochromocytoma      & Malignant            \\
                                        &                                     &                       & pheochromocytoma     \\
                                        &                                     & Ganglioneuroma        & Neuroblastoma        \\
                                        &                                     &                       & Ganglioneuroblastoma \\
        \midrule
        \multirow{4}{*}{Secondary}      & \multirow{4}{*}{No specific origin} & Myelolipoma           & Metastases           \\
                                        &                                     & Cysts                 & Lymphoma             \\
                                        &                                     & Lipoma                &                      \\
                                        &                                     & Hemorrhage            &                      \\
        \midrule
        \multirow{3}{*}{Other entities} & \multirow{2}{*}{Any cell origin/}   & Incidentalomas        & Colision tumours     \\
                                        &                                     &                       & (adenoma+metastases) \\
                                        & \multirow{1}{*}{either primary}     & Colision tumours      &                      \\
                                        & \multirow{1}{*}{or secondary}       & (adenoma+myelolipoma) &

        \\\bottomrule
    \end{tabular}
    \caption{Classification of the most common adrenal lesions in terms of their
        etiology and cell of origin.  Adapted from \cite{Panda2015}.}
    \label{tab:lesions}
\end{table}

In general, non-functional lesions do not require any treatment, therefore it is
crucial to differentiate between adenomas (typical non-functional lesions) and
non-adenomas \cite{Platzek2019}. Adrenal adenomas have less than 1 cm in
diameter, usually and they can be lipid-rich or lipid-poor \cite{Panda2015}.
About 70-80 \% of the adenomas are lipid-rich in contrast with the malignant
lesions \cite{Platzek2019}. This results in a 20-30 \% overlap between adenomas
and malignant lesions in terms of intracytoplasmic lipid content
\cite{Israel2004}.

Functional lesions can cause endocrines syndromes, such as
Conn and Cushing syndrome. Cushing syndrome or hypercortisolism is caused by
elevated cortisol values and is associated with adrenal adenomas. However,
adrenocortical carcinomas or pheochromocytomas can also cause Cushing syndrome.
This syndrome is defined by symptoms such as obesity, rounded face, abnormal
skin pigmentation, muscle weakness, hypertension, diabetes, and others. On the
other hand, the Conn syndrome or primary aldosteronism is related to the
excessive production of aldosterone. The most common symptoms of this syndrome
are sodium retention, plasma renin suppression, hypertension, cardiovascular
damage, and increased potassium excretion. Like the Conn syndrome, this syndrome
is commonly caused by adrenal adenomas.  In opposition to the above-mentioned
syndromes, the Addison disease is caused by adrenal insufficiency, which can be
caused by malignant lesions. Patients with this disease can experience weight
loss, weakness, fatigue, gastrointestinal upset, orthostatic hypotension, and
abnormal skin pigmentation. These symptoms can progress to dehydration, shock,
hyperkalaemia (high potassium) and hyponatremia (high sodium) when entering
acute adrenal insufficiency \cite{Wang2018}.

The adrenal gland has two different regions: the adrenal cortex and the adrenal
medulla. The adrenal cortex is the outer region of the gland and is derived from
neural crest cells. The cortex itself comprises three zones: zona glomerosa, the
zona fasciculata, and zona reticularis, each responsible for producing a
different set of hormones. The adrenal cortex secretes hormones relevant to the
regulation of long-term stress response, blood pressure and blood volume,
nutrient uptake and storage, fluid and electrolyte balance, and inflammation.
For example, this region is responsible for producing cortisol, corticosterone,
and cortisone, which increase blood glucose levels, and aldosterone production,
which increases the sodium level in blood \cite{open}.

The adrenal medulla is the inner region of the gland, and it is a neuroendocrine
tissue composed of postganglionic sympathetic nervous system (SNS) neurons. This
region is responsible for the production of epinephrine and norepinephrine,
which mediate the short-term stress or the fight-or-flight response. This
response has the goal of prepare the body for extreme physical exertion
\cite{open}.

\subsection{Adrenals Imaging}

Structural medical imaging techniques are decisive to detect and characterize
adrenal lesions and complementary to functional imaging and endocrine evaluation
in the assessment of functional lesions. Imaging techniques can also rule out
invasive interventions. The most used imaging techniques to evaluate the adrenal
glands are Computed Tomography (CT) and Magnetic Resonance Imaging (MRI). The
Ultrasonography (USG) despite being a common method to assess abdominal
pathologies, is not a good method to perceive retroperitoneal (back of the
peritoneum) structures like the adrenals \cite{Panda2015}. Both Figures <ref FIG
2> and <REF FIG 3> show the V and Y-shaped normal glands. Figure <ref FIG 2> is
an axial contrast-enhanced CT image in the arterial phase where both glands are
enhanced due to the high retroperitoneal fat content. Figure <REF FIG 3> is a
coronal MR Chemical Shift Image (CSI) out-of-phase showing normal adrenal glands
also.

\begin{figure}[htbp]
    \centering
    \includegraphics[scale=0.15 ]{figures/retro_peritoneum.jpg}
    \caption{Peritoneum representation. Adrenals are an example of a retroperitoneal organ. From \cite{retroimg} }
    \label{fig: retro_per}
\end{figure}

\begin{figure}
    \centering
    \begin{subfigure}[b]{0.45\textwidth}
        \centering
        \includegraphics[width=\textwidth]{figures/CT_adrenal_enhanced.png}
        \caption{Adrenal glands in a contrast-enhanced CT axial slice in arterial phase. Due to the high level of retroperitoneal fat both glands are enhanced in this image slice. Reprinted from \cite{Panda2015}}
        \label{fig:adrenal_ct}
    \end{subfigure}
    \hfill
    \begin{subfigure}[b]{0.45\textwidth}
        \centering
        \includegraphics[width=\textwidth]{figures/MRI_adrenal.png}
        \caption{Adrenal glands in a MR CSI coronal slice. Both glands have a intermediate signal intensity. Reprinted from \cite{Panda2015} }
        \label{fig:adrenal_mri}
    \end{subfigure}
    \caption{Normal adrenal glands in CT and MR slices. The arrows indicate the localization of the glands.}
    \label{fig:adrenal_glands_normal}
\end{figure}

Lipid-rich adenomas can be easily detected using unenhanced CT (less than 10 HU)
\cite{Panda2015} or CSI \cite{Platzek2019}. However, lipid-poor adenomas cannot
be correctly characterised by  \cite{Israel2004}. In these cases,
CSI presents itself as a better solution because of its improved sensitivity to
low levels of lipid content and therefore it can detect 62-67\% of the adenomas
uncharacterised by CT \cite{Israel2004}. CSI is a fat-suppression technique that
originates two sets of images: in-phase (IP) and out-of-phase (OP) images. In OP
images the signal is the difference between the signals of water and fat
molecules. In IP images the signal of both water and fat is added. Thus, there
is a significant suppression of signal from IP to OP images in lipid-rich
lesions \cite{Jahanvi2021}. OP images are characterised by the so-called India
ink artefact that is a signal void in the margins of fatty and normal tissues
\cite{Jahanvi2021}, creating a darker boundary in lipid-rich lesions like most
adenomas. \cite{Platzek2019} performed a metanalysis with 1280 lesions (859
adenomas, 421 non-adenomas) and documented a sensitivity of 94\% and specificity
of 95\% detecting adrenal adenomas. The same study states the importance of
using Dynamic Contrast-Enhanced (DCE) MRI to improve the detection performance
of lipid-poor adenomas. Adenomas present a signal increase in the arterial phase
and a rapid washout \cite{Chung2001}.

\begin{figure}
    \centering
    \begin{subfigure}[b]{0.45\textwidth}
        \centering
        \includegraphics[width=\textwidth]{figures/S016_I027_93033860_Axial_GR-['SS','SP','SK']-UNKNOWN_no_60_70.png}
        \caption{T1-weighted out-of-phase axial slice with a lipid-rich adenoma.}
        \label{fig:adenoma_lr_OP}
    \end{subfigure}
    \hfill
    \begin{subfigure}[b]{0.45\textwidth}
        \centering
        \includegraphics[width=\textwidth]{figures/S016_I028_93033860_Axial_GR-['SS','SP','SK']-UNKNOWN_no_60_70.png}
        \caption{T1-weighted in-phase axial slice with a lipid-rich adenoma.}
        \label{fig:adenoma_lr_IP}
    \end{subfigure}
    \vfill
    \begin{subfigure}[b]{0.45\textwidth}
        \centering
        \includegraphics[width=\textwidth]{figures/S003_I023_93036417_Axial_GR-['SS','SP','SK']-UNKNOWN_no_70_80.png}
        \caption{T1-weighted out-of-phase axial slice with a lipid-poor adenoma.}
        \label{fig:adenoma_lp_OP}
    \end{subfigure}
    \hfill
    \begin{subfigure}[b]{0.45\textwidth}
        \centering
        \includegraphics[width=\textwidth]{figures/S003_I024_93036417_Axial_GR-['SS','SP','SK']-UNKNOWN_no_70_80.png}
        \caption{T1-weighted in-phase axial slice with a lipid-poor adenoma.}
        \label{fig:adenoma_lp_IP}
    \end{subfigure}
    \caption{Adrenal adenomas in axial MR CSI. The red rectangles surround the adenomas. Lipid-rich adenomas have a much greater intensity difference between in-phase and out-of-phase images.   }
    \label{fig:adenomas}
\end{figure}

\section{Differential diagnosis of adrenal lesions - current medical approach}

\subsection{Current procedure}

\subsection{Main limitations and setbacks}

\section{Differential diagnosis of adrenal lesions - machine learning approach}

\subsection{Research details}

The studies analysed in this paper were selected accordingly in the following PICO criteria:

\begin{enumerate}
    \item[] \textbf{P (patients) }– patients with adrenal lesions.
    \item[] \textbf{I (interventions) }– machine learning (including deep learning) modelling.
    \item[] \textbf{C (comparison) }– standard of care imaging including Computed Tomography (CT) and Magnetic Imaging Resonance (MRI).
    \item[] \textbf{O (outcome) }– lesion differentiation (benign/malign and subtyping) and lesions detection.
\end{enumerate}

The studies were searched in two databases: PubMed and Web of Science in
September 2022. The search was made with the following research string: (adrenal
or suprarenal) AND (CT OR "computed tomography" OR MRI OR "magnetic resonance
imaging" OR "MRI scan" OR "nuclear magnetic resonance" OR "magnetic resonance"
OR NMR) AND ("deep learning" OR "convolutional networks" OR CNN OR "neural
networks" OR convolutional OR DNN OR SVM OR "Support vector machine" OR
"decision tree" OR "machine learning"). Studies that were: (a) reviews, (b) not
written in English, (c) did not report a modelling method, (d) did not have the
full text available were excluded from this research.

The research resulted in 23 studies that were divided in 3 groups related with
their object of study:
\begin{enumerate}
    \item[] \textbf{Group A}: contains all the studies that focus on the
        differentiation between adrenal adenomas and other adrenal lesions.
    \item[] \textbf{Group B}: contains all the studies that target the
        differentiation between benign and malign adrenal lesions.
    \item[] \textbf{Group C}: contains the remaining studies that did not fit
        any of the above categories.
\end{enumerate}
Table \ref{tab:sota_sum} shows the distribution of the selected studies in terms
of their group, image modality and model type. Overall, most of the studies
adopt traditional machine learning models to classify imaging features
(radiomics) from CT images to distinguish adenomas from other types of adrenal
lesions, eg. Metastases, pheochromocytomas.

\begin{table}[]
    \centering
    \begin{tabular}{cccccccc}\toprule
        \multirow{2}{*}{\textbf{Group}} & \multicolumn{3}{c}{\textbf{Image Modality}} & \multicolumn{3}{c}{\textbf{Model Type}} & \multirow{2}{*}{\textbf{Total}}
        \\\cmidrule(lr){2-4} \cmidrule(lr){5-7}
                                        & \textbf{MRI}                                & \textbf{CT}                             & \textbf{MRI+CT}                 & \textbf{ML} & \textbf{DL} & \textbf{ML+DL}      \\\midrule
        A - Adenomas vs other lesions   & 4                                           & 7                                       & 1                               & 11          & 1           & 0              & 12 \\
        B - Benign vs Malign            & 2                                           & 5                                       & 0                               & 7           & 0           & 0              & 7  \\
        C - Other                       & 1                                           & 3                                       & 0                               & 2           & 1           & 1              & 4  \\
        Sum                             & 7                                           & 15                                      & 1                               & 20          & 2           & 1              & 23 \\
        \bottomrule
    \end{tabular}
    \caption{Studies distribution per group. CT: Computed Tomography; MRI: Magnetic Resonance Imaging; ML: Traditional Machine Learning; DL: Deep Learning.}
    \label{tab:sota_sum}
\end{table}

Radiomics is the extraction of quantitative features from medical images. Before the feature extraction, there are two relevant
steps: image acquisition and segmentation of the region of interest (ROI). Any type of medical images can be used in radiomics,
such as Positron Emission Tomography (PET), MRI, or CT. The ROI segmentation is necessary to limit the amount of data that needs
to be processed in order to extract features. This process can be completely manual, which is the gold standard, where a specialist
selects the ROI. Manual segmentation is a time-consuming task that significantly depends on the skill of the operator. There are
fully automatic methods for ROI segmentation however, they can fail in difficult cases, such as, lesions with indistinct borders
and are highly dependent on the quality of the image. For that reason, the usage of semi-automatic methods is preferable.
These methods have minimal user interaction (seed identification or manual correction). The extracted quantitative features aim
to describe the complexity of the individual region of interest. Ordinarily, these features are divided into 4 categories:
\begin{enumerate}
    \item Shape-Based Features: numeric information respecting geometrics characteristics, like shape and size.
    \item First-Order Statistics: distribution of voxel values without spatial information, generally histogram-based.
    \item Second-Order Statistics: “texture” features, focus on the spatial relationships between voxels with similar grey levels.
    \item High-Order Statistics: usage of filters to extract patterns from the images. From the resultant images, first and second-order features are extracted.
\end{enumerate}

The most relevant radiomic features to the task in hand are selected using statistical approaches or machine learning \cite{Zhang2022}. Then, these features are used as the input of ML models to classify the region of interest. This type of workflow is widespread, appearing in 87\% (20/23) of the analysed studies.
Traditional machine learning models like K-means, Support Vector Machine (SVM), Logistic Regression (LogReg), or Random Forest (RanFor), are frequently used to classify radiomic features of regions of interest and have achieved high performance in different anatomical regions \cite{Zhang2022, Wagner2021}. Deep Learning models, such as, Convolutional Neural Networks (CNNs), have been often applied to medical images from several anatomical regions with promising results \cite{Anaya-Isaza2021}. Despite that, only 13\% of the studies presented in this paper report the application of DL models to adrenal images. DL models are different from ML models mostly because they demand bigger annotated data sets and they do not rely on the feature extraction step (all features are automatically extracted and classified by the model).

In the next sections each group of papers will be analysed in detail, exploring their common and contrasting aspects. For each group the analysis will be focused on the utilized data sets, models, and the obtained results.

\subsection{Group A - Adenomas vs other lesions}

Group A comprises all studies that focus on the differentiation between adenomas
and other lesions. This group corresponds to 52\% of all analysed studies.
Tables \ref{tab:data_A}, \ref{tab:model_A} and \ref{tab:res_A} present an
overview of the data sets, models, and results, respectively, for each study
within group A. \cite{Schieda2017,Tu2018,Tu2020} have data sets with adenomas
and metastases. \cite{Yi20181,Yi2018, Liu2022,Liu2021} have patients with
adenomas and pheochromocytomas. \cite{Elmohr2019, Torresan2021, Ho2019} have
compared adenomas and carcinomas. \cite{Tu2020, Yi2018, Yi20181, Liu2022} have
considered only lipid-poor adenomas for their adenoma dataset. Finally,
\cite{Romeo2018} data set has 3 classes: lipid-poor adenomas, lipid-rich
adenomas and nonadenomas and \cite{Kusunoki2022} performs a general
differentiation between adenomas and nonadenomas. In terms of the imaging
modality utilised by the studies, there is slight advantage to CT, as it is used
in 8 studies, mostly with and without contrast enhanced images. The studies that
analyse MRI images focus on chemical shift images as they pose an advantage in
the detection of the lipid content of the lesions. The sample size of the
studies ranges from dozens to hundreds of lesions, which is closely related to
the applied inclusion criteria and the initial sample size. For example,
\cite{Torresan2021} had a database of 336 patients however only 19 met the
inclusion criteria, each with one lesion.

All studies,but one, performed lesion classification with ML models using
radiomic features. The most common model is logistic regression (LogReg),
followed by support vector machine (SVM) and decision tree-based models. In
every study, the region of interest (ROI) was manually selected by experts. The
extraction of first and second-order statistics is a common practice, but only 3
studies extracted shape-based features, and none extracted higher-order
statistics. The only study that has implemented a DL model has performed ROI
(selected, cropped and labelled by experts) classification with a deep
convolution neural network \cite{Kusunoki2022}. They report the usage of
augmentation techniques such as rotations and horizontal flips.
\cite{Torresan2021} was the only study to implement an unsupervised model.

In \cite{Ho2019} the data set consists of lipid-poor adenomas and carcinomas in
both MRI and CT images. The objective of this work was to compare the different
image modalities using the same machine learning approach. The authors have
reported only the Area Under the Receiving Operating Characteristic Curve (AUC).
The value presented in \ref{tab:res_A} refers to the best result, using CE-CT
images, with MRI images the value decreases to 58 \%.
Both studies by Yi et al \cite{Yi2018, Yi20181} implemented a logistic regression model with the same radiomic features
but using different CT images and achieved impressive results. \cite{Yi2018} adds clinical features
such as, the existence of necrosis or calcification and lesion dimensions, to
the radiomic features. Also, to improve feature selection the authors used the
Least Absolute Shrinkage and Selection Operator (LASSO).



\begin{table}[]
    \centering
    \begin{tabular}{ccccc}\toprule
        \multirow{2}{*}{\textbf{Reference}} & \multirow{2}{*}{\textbf{Image Modality}} & \multicolumn{3}{c}{\textbf{Sample Size (lesions)}}
        \\\cmidrule(lr){3-5}
                                            &                                          & \textbf{Total}                                     & \textbf{Adenomas} & \textbf{Other} \\\midrule
        \cite{Tu2018}                       & U-CT                                     & 76                                                 & 36                & 40             \\
        \cite{Yi20181}                      & U/CE-CT                                  & 110                                                & 80                & 30             \\
        \cite{Yi2018}                       & U-CT                                     & 265                                                & 181               & 84             \\
        \cite{Elmohr2019}                   & CE-CT                                    & 54                                                 & 25                & 29             \\
        \cite{Torresan2021}                 & U/CE-CT                                  & 19                                                 & 9                 & 10             \\
        \cite{Kusunoki2022}                 & U/CE-CT                                  & 115                                                & 83                & 32             \\
        \cite{Liu2022}                      & U/CE-CT                                  & 280                                                & 188               & 92             \\
        \cite{Ho2019}                       & U/CE-CT; T1W-OP/IP MRI                   & 23                                                 & 15                & 8              \\
        \cite{Liu2021}                      & T1W-OP/IP; T2W MRI                       & 60                                                 & 40                & 20             \\
        \cite{Schieda2017}                  & T1W-OP/IP; T2W MRI                       & 44                                                 & 29                & 15             \\
        \cite{Tu2020}                       & T1W-OP/IP; T2W MRI                       & 63                                                 & 23                & 40             \\
        \cite{Romeo2018}                    & T1W-OP/IP; T2W MRI                       & 60                                                 & 40                & 20
        \\\bottomrule
    \end{tabular}
    \caption{Dataset Details for each article in the Group A. CT: Computed Tomography; U: Unenhanced; CE: Contrast Enhanced; MRI: Magnetic Resonance Imaging; OP: Out-of-phase; IP: In-phase; T1W: T1-weighted; T2W: T2-weighted.}
    \label{tab:data_A}
\end{table}

\begin{table}[]
    \centering
    \begin{tabular}{cccccc}\toprule
        \multirow{2}{*}{\textbf{Reference}} & \multirow{2}{*}{\textbf{Type}} & \multirow{2}{*}{\textbf{Classification Model}} & \multirow{2}{*}{\textbf{ROI}} & \multirow{2}{*}{\textbf{Features}} \\
        \\\midrule
        \cite{Tu2018}                       & ML                             & LogReg                                         & Manual                        & $1^{st}$                           \\
        \cite{Yi20181}                      & ML                             &
        LogReg                              & Manual                         & $1^{st}$, $2^{nd}$, higher                                                                                          \\
        \cite{Yi2018}                       & ML                             &
        LogReg                              & Manual
                                            & $1^{st}$, $2^{nd}$, higher                                                                                                                           \\
        \cite{Elmohr2019}                   & ML                             & RanFor; LogReg                                 & Manual                        & $1^{st}$, $2^{nd}$, shape          \\
        \cite{Torresan2021}                 & ML                             & K-Means                                        & Manual                        & $1^{st}$, $2^{nd}$                 \\
        \cite{Kusunoki2022}                 & DL                             & DCNN                                           & Manual                        & -                                  \\
        \cite{Liu2022}                      & ML                             & LinReg; SVM; RanFor                            & Manual                        & $1^{st}$; clinical                 \\
        \cite{Ho2019}                       & ML                             & LogReg                                         & Manual                        & $1^{st}$, $2^{nd}$, shape          \\
        \cite{Liu2021}                      & ML                             & SVM                                            & Manual                        & $1^{st}$                           \\
        \cite{Schieda2017}                  & ML                             & LogReg                                         & Manual                        & $1^{st}$                           \\
        \cite{Tu2020}                       & ML                             & LogReg                                         & Manual                        & $1^{st}$, shape                    \\
        \cite{Romeo2018}                    & ML                             & DecTre                                         & Manual                        & $1^{st}$, $2^{nd}$                 \\
        \bottomrule
    \end{tabular}
    \caption{Modelling Details  for each article in the Group A. ML: Traditional Machine Learning models; DL: Deep Learning models; LogReg: Logistic Regression; LASSO: Least Absolute Shrinkage and Selection Operator; DecTre: Decision Tree; RanFor: Random Forest; PCA: Principal Components Analysis; SVM: Support Vector Machine; $1^{st}$, $2^{nd}$, higher: first, second, higher order statistics, respectively; shape: shape-based features.}
    \label{tab:model_A}
\end{table}

\begin{table}[]
    \centering
    \begin{tabular}{cccccc}\toprule
        \multirow{2}{*}{\textbf{Reference}} & \multirow{2}{*}{\textbf{Specificity - \%}} & \multirow{2}{*}{\textbf{Sensitivity - \%}} & \multirow{2}{*}{\textbf{Accuracy - \%}} & \multirow{2}{*}{\textbf{AUC - \%}} \\
        \\\midrule
        \cite{Tu2018}                       & 75.0                                       & 47.5                                       & 60.5                                    & 65.0                               \\
        \cite{Yi20181}                      & 97.5                                       & 86.2                                       & 94.4                                    & 95.2                               \\
        \cite{Yi2018}                       & 90.3                                       & 95.5                                       & 92.0                                    & 95.7                               \\
        \cite{Elmohr2019}                   & 83.0                                       & 81.0                                       & 82.0                                    & 89.0                               \\
        \cite{Torresan2021}                 & 90.0                                       & 87.5                                       & 88.9                                    & -                                  \\
        \cite{Kusunoki2022}                 & 96.0                                       & 87.0                                       & 94.0                                    & -                                  \\
        \cite{Liu2022}                      & 86.6                                       & 89.2                                       & 87.5                                    & -                                  \\
        \cite{Ho2019}                       & -                                          & -                                          & 80.0                                    & -                                  \\
        \cite{Liu2021}                      & -                                          & -                                          & 85.0                                    & 91.7                               \\
        \cite{Schieda2017}                  & 86.2                                       & 93.3                                       & 88.6                                    & 97.0                               \\
        \cite{Tu2020}                       & 100                                        & 75.0                                       & 84.1                                    & -                                  \\
        \cite{Romeo2018}                    & -                                          & -                                          & 80.0                                    & -                                  \\
        \bottomrule
    \end{tabular}
    \caption{Model metrics for each article in the Group A. AUC: Area Under the Receiving Operating Characteristic Curve.}
    \label{tab:res_A}
\end{table}

\subsection{Group B - Malign vs benign lesions}

Group B consists of studies that aim at the differentiation between benign and
malignant lesions. This group includes 30\% of the studies. Tables
\ref{tab:data_B}, \ref{tab:model_B} and \ref{tab:res_B} display an overview of
the data sets, models, and results, respectively, for each study within group B.
In terms of the image modalities in the data sets, this group is similar to
group A. There are more studies that use CT images and all of them, expect one,
use CE-CT. Studies that analyse MRI data sets have both T1W chemical shift
images and T2W images. \cite{Shoemaker2018} has the largest data set of all the
analysed studies. Group B does not comprise any study with a DL model, however
there are two studies that use neural networks. \cite{Koyuncu2019} experimented
several optimisation algorithms for neural networks and achieved the best
results using the Bounded Particle Swarm Optimisation algorithm
\footnote{ \href{https://link.springer.com/chapter/10.1007/978-3-319-93025-1\_2}{https://link.springer.com/chapter/10.1007/978-3-319-93025-1\_2}}.
In \cite{Barstugan2020} a SVM was implemented to perform binary classification,
however, the authors also used a NN to perform type characterisation. The
authors divided the dataset in 4 classes, each with one type of lesion, 3 benign
(adenoma, cyst and lipoma) and 1 malign (metastasis). For both workflows, ROI
selection was made using manual and semi-automatic segmentation, and the same
radiomic features were used. The results presented in Table \ref{tab:res_B}
refer to the binary classification, and they were the best results of this
group. For the multiclass classification the results were poor, despite the high
values of specificity and accuracy, 96.2 \% and 93.2 \%, respectively, the
sensibility is extremely low, 59.6 \%, which can be explained by the high number
of class and the lack of balance in the data set. In this group all the models
are supervised learning models.


\begin{table}[]
    \centering
    \begin{tabular}{ccccc}\toprule
        \multirow{2}{*}{\textbf{Reference}} & \multirow{2}{*}{\textbf{Image Modality}} & \multicolumn{3}{c}{\textbf{Sample Size (lesions)}}
        \\\cmidrule(lr){3-5}
                                            &                                          & \textbf{Total}                                     & \textbf{Benign} & \textbf{Malign} \\\midrule
        \cite{Shoemaker2018}                & U-CT                                     & 377                                                & 182             & 195             \\
        \cite{Koyuncu2019}                  & CE-CT                                    & 114                                                & 90              & 24              \\
        \cite{Li2019}                       & U/CE-CT                                  & 210                                                & 114             & 96              \\
        \cite{Andersen2021}                 & CE-CT                                    & 160                                                & 89              & 71              \\
        \cite{Moawad2021}                   & U/CE-CT                                  & 40                                                 & 21              & 19              \\
        \cite{Barstugan2020}                & T1W-OP/IP; T2W MRI                       & 114                                                & 9               & 105             \\
        \cite{Stanzione2021}                & T1W-OP/IP; T2W MRI                       & 55                                                 & 37              & 18              \\
        \bottomrule
    \end{tabular}
    \caption{Dataset Details for each article in the Group B. CT: Computed Tomography; U: Unenhanced; CE: Contrast Enhanced; MRI: Magnetic Resonance Imaging; OP: Out-of-phase; IP: In-phase; T1W: T1-weighted; T2W: T2-weighted.}
    \label{tab:data_B}
\end{table}

\begin{table}[]
    \centering
    \begin{tabular}{ccccc}\toprule
        \multirow{2}{*}{\textbf{Reference}} & \multirow{2}{*}{\textbf{Type}} & \multirow{2}{*}{\textbf{Classification Model}} & \multirow{2}{*}{\textbf{ROI}} & \multirow{2}{*}{\textbf{Features}} \\
        \\ \midrule
        \cite{Shoemaker2018}                & ML                             & LogReg                                         & -                             & $1^{st}$, $2^{nd}$                 \\
        \cite{Koyuncu2019}                  & ML                             & NN                                             & Semi-auto                     & $1^{st}$, $2^{nd}$, higher, shape  \\
        \cite{Li2019}                       & ML                             & BayCla                                         & Semi-auto                     & $2^{nd}$                           \\
        \cite{Andersen2021}                 & ML                             & LogReg                                         & Semi-auto                     & $1^{st}$, higher                   \\
        \cite{Moawad2021}                   & ML                             & RanFor                                         & Manual                        & $1^{st}$, $2^{nd}$, higher, shape  \\
        \cite{Barstugan2020}                & ML                             &
        SVM                                 & Manual; Semi-auto
                                            & $2^{nd}$, higher                                                                                                                                     \\
        \cite{Stanzione2021}                & ML                             & ExTre                                          & Manual                        & $1^{st}$, $2^{nd}$, higher, shape  \\
        \bottomrule
    \end{tabular}
    \caption{Modelling Details for each article in the Group B. ML: Traditional Machine Learning models; DL: Deep Learning models; LogReg: Logistic Regression; BayCla: Bayesian Classifier; NN: Neural Network; ExTre: Extra Trees Classifier.
        RanFor: Random Forest; SVM: Support Vector Machine; $1^{st}$, $2^{nd}$, higher: first, second, higher order statistics, respectively; shape: shape-based features.}
    \label{tab:model_B}
\end{table}

\begin{table}[]
    \centering
    \begin{tabular}{ccccc}\toprule
        \multirow{2}{*}{\textbf{Reference}} & \multirow{2}{*}{\textbf{Specificity - \%}} & \multirow{2}{*}{\textbf{Sensitivity - \%}} & \multirow{2}{*}{\textbf{Accuracy - \%}} & \multirow{2}{*}{\textbf{AUC - \%}} \\
        \\\midrule
        \cite{Shoemaker2018}                & -                                          & -                                          & -                                       & 78.0                               \\
        \cite{Koyuncu2019}                  & 82.2                                       & 75.0                                       & 80.7                                    & 78.6                               \\
        \cite{Li2019}                       & 67.5                                       & 94.8                                       & 80.0                                    & -                                  \\
        \cite{Andersen2021}                 & 77.0                                       & 58.0                                       & 68.0                                    & 73.0                               \\
        \cite{Moawad2021}                   & 71.4                                       & 84.2                                       & 77.5                                    & 85.1                               \\
        \cite{Barstugan2020}                & 90.0                                       & 99.2                                       & 98.4                                    & -                                  \\
        \cite{Stanzione2021}                & -                                          & -                                          & 91.0                                    & 97.0                               \\
        \bottomrule
    \end{tabular}
    \caption{Model metrics for each article in the Group B. AUC: Area Under the Receiving Operating Characteristic Curve.}
    \label{tab:res_B}
\end{table}

\subsection{Group C}

Group C consists of the remaining studies that did not fit any of prior defined
groups.  This group includes 18\% of the studies. Tables \ref{tab:data_B},
\ref{tab:model_B} and \ref{tab:res_B} display an overview of the data sets,
models, and results, respectively, for each study inside group C. In this group
are 4 studies with distinct objectives. \cite{Bi2017} implements a fully
convolutional neural network for lesion detection using a small dataset of U-CT
images. \cite{Bi2022} creates a machine learning pipeline where the CNN
embedding is used as input of an SVM to execute multiclass classification with a
data set of CE-CT images. The data set had 5 classes: carcinoma, non-functional
adenoma, ganglioneuroma, myelolipoma and pheochromocytoma.\cite{Kong2022} aimed
at the differentiation between pheochromocytomas and non-pheochromocytomas with
a T2W MRI dataset and a logistic regression model. \cite{Zheng2020} also used
logistic regression but to characterize adenomas with a CT dataset.
\begin{table}[]
    \centering
    \begin{tabular}{ccccc}\toprule
        \multirow{2}{*}{\textbf{Reference}} & \multirow{2}{*}{\textbf{Image Modality}} & \multirow{2}{*}{\textbf{Sample size}} & \multirow{2}{*}{\textbf{Task}} \\
        \\ \midrule
        \cite{Bi2017}                       & U-CT                                     & 38                                    & Lesion Detection               \\
        \cite{Bi2022}                       & CE-CT                                    & 229                                   & Multiclass Classification      \\
        \cite{Kong2022}                     & T2W-MRI                                  & 305                                   & pheo vs non-pheo               \\
        \cite{Zheng2020}                    & U/CE-CT                                  & 83                                    & Adenoma subtyping              \\
        \bottomrule
    \end{tabular}
    \caption{Dataset Details for each article in the Group C. CT: Computed Tomography; U: Unenhanced; CE: Contrast Enhanced; MRI: Magnetic Resonance Imaging; OP: Out-of-phase; IP: In-phase; T1W: T1-weighted; T2W: T2-weighted. Pheo: pheochromocytomas}
    \label{tab:data_C}
\end{table}

\begin{table}[]
    \centering
    \begin{tabular}{ccccc}\toprule
        \multirow{2}{*}{\textbf{Reference}} & \multirow{2}{*}{\textbf{Type}} & \multirow{2}{*}{\textbf{Model}} & \multirow{2}{*}{\textbf{ROI}} & \multirow{2}{*}{\textbf{Features}} \\
        \\ \midrule
        \cite{Bi2017}                       & DL                             & FCN                             & Manual                        & -                                  \\
        \cite{Bi2022}                       & DL + ML                        & CNN + SVM                       & Manual                        & CNN embedding                      \\
        \cite{Kong2022}                     & ML                             & LogReg                          & Semi-Auto                     & $1^{st}$, $2^{nd}$, higher, shape  \\
        \cite{Zheng2020}                    & ML                             & LogReg                          & Manual                        & $1^{st}$, higher, shape            \\
        \bottomrule
    \end{tabular}
    \caption{Modelling Details for each article in the Group C. ML: Traditional Machine Learning models; DL: Deep Learning models; LogReg: Logistic Regression;
        BSGC: Bayesian Spatial Gaussian Classifiers; CNN: Convolutional Neural Network; SVM: Support Vector Machine;
        $1^{st}$, $2^{nd}$, higher: first, second, higher order statistics, respectively; shape: shape-based features.}
    \label{tab:model_C}
\end{table}

\begin{table}[]
    \centering
    \begin{tabular}{ccccc}\toprule
        \multirow{2}{*}{\textbf{Reference}} & \multirow{2}{*}{\textbf{Specificity - \%}} & \multirow{2}{*}{\textbf{Sensitivity - \%}} & \multirow{2}{*}{\textbf{Accuracy - \%}} & \multirow{2}{*}{\textbf{AUC - \%}} \\
        \\\midrule
        \cite{Bi2017}                       & -                                          & 76.29                                      & -                                       & -                                  \\
        \cite{Bi2022}                       & 95.9                                       & 83.7                                       & 85.2                                    & -                                  \\
        \cite{Kong2022}                     & 75.0                                       & 85.7                                       & 84.0                                    & 90.6                               \\
        \cite{Zheng2020}                    & 92.8                                       & 91.5                                       & 92.2                                    & 90.2                               \\
        \bottomrule
    \end{tabular}
    \caption{Model metrics for each article in the Group C. AUC: Area Under the Receiving Operating Characteristic Curve.}
    \label{tab:res_C}
\end{table}

% \bibliographystyle{unsrt}
% \bibliography{adrenal}
\printbibliography

\end{document}